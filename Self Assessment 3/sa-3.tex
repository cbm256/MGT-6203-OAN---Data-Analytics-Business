\documentclass[]{article}
\usepackage{lmodern}
\usepackage{amssymb,amsmath}
\usepackage{ifxetex,ifluatex}
\usepackage{fixltx2e} % provides \textsubscript
\ifnum 0\ifxetex 1\fi\ifluatex 1\fi=0 % if pdftex
  \usepackage[T1]{fontenc}
  \usepackage[utf8]{inputenc}
\else % if luatex or xelatex
  \ifxetex
    \usepackage{mathspec}
  \else
    \usepackage{fontspec}
  \fi
  \defaultfontfeatures{Ligatures=TeX,Scale=MatchLowercase}
\fi
% use upquote if available, for straight quotes in verbatim environments
\IfFileExists{upquote.sty}{\usepackage{upquote}}{}
% use microtype if available
\IfFileExists{microtype.sty}{%
\usepackage{microtype}
\UseMicrotypeSet[protrusion]{basicmath} % disable protrusion for tt fonts
}{}
\usepackage[margin=1in]{geometry}
\usepackage{hyperref}
\hypersetup{unicode=true,
            pdftitle={WEEK 7: SELF ASSESSMENT 3 SOLUTIONS - Shrikanth Mahale},
            pdfborder={0 0 0},
            breaklinks=true}
\urlstyle{same}  % don't use monospace font for urls
\usepackage{color}
\usepackage{fancyvrb}
\newcommand{\VerbBar}{|}
\newcommand{\VERB}{\Verb[commandchars=\\\{\}]}
\DefineVerbatimEnvironment{Highlighting}{Verbatim}{commandchars=\\\{\}}
% Add ',fontsize=\small' for more characters per line
\usepackage{framed}
\definecolor{shadecolor}{RGB}{248,248,248}
\newenvironment{Shaded}{\begin{snugshade}}{\end{snugshade}}
\newcommand{\KeywordTok}[1]{\textcolor[rgb]{0.13,0.29,0.53}{\textbf{{#1}}}}
\newcommand{\DataTypeTok}[1]{\textcolor[rgb]{0.13,0.29,0.53}{{#1}}}
\newcommand{\DecValTok}[1]{\textcolor[rgb]{0.00,0.00,0.81}{{#1}}}
\newcommand{\BaseNTok}[1]{\textcolor[rgb]{0.00,0.00,0.81}{{#1}}}
\newcommand{\FloatTok}[1]{\textcolor[rgb]{0.00,0.00,0.81}{{#1}}}
\newcommand{\ConstantTok}[1]{\textcolor[rgb]{0.00,0.00,0.00}{{#1}}}
\newcommand{\CharTok}[1]{\textcolor[rgb]{0.31,0.60,0.02}{{#1}}}
\newcommand{\SpecialCharTok}[1]{\textcolor[rgb]{0.00,0.00,0.00}{{#1}}}
\newcommand{\StringTok}[1]{\textcolor[rgb]{0.31,0.60,0.02}{{#1}}}
\newcommand{\VerbatimStringTok}[1]{\textcolor[rgb]{0.31,0.60,0.02}{{#1}}}
\newcommand{\SpecialStringTok}[1]{\textcolor[rgb]{0.31,0.60,0.02}{{#1}}}
\newcommand{\ImportTok}[1]{{#1}}
\newcommand{\CommentTok}[1]{\textcolor[rgb]{0.56,0.35,0.01}{\textit{{#1}}}}
\newcommand{\DocumentationTok}[1]{\textcolor[rgb]{0.56,0.35,0.01}{\textbf{\textit{{#1}}}}}
\newcommand{\AnnotationTok}[1]{\textcolor[rgb]{0.56,0.35,0.01}{\textbf{\textit{{#1}}}}}
\newcommand{\CommentVarTok}[1]{\textcolor[rgb]{0.56,0.35,0.01}{\textbf{\textit{{#1}}}}}
\newcommand{\OtherTok}[1]{\textcolor[rgb]{0.56,0.35,0.01}{{#1}}}
\newcommand{\FunctionTok}[1]{\textcolor[rgb]{0.00,0.00,0.00}{{#1}}}
\newcommand{\VariableTok}[1]{\textcolor[rgb]{0.00,0.00,0.00}{{#1}}}
\newcommand{\ControlFlowTok}[1]{\textcolor[rgb]{0.13,0.29,0.53}{\textbf{{#1}}}}
\newcommand{\OperatorTok}[1]{\textcolor[rgb]{0.81,0.36,0.00}{\textbf{{#1}}}}
\newcommand{\BuiltInTok}[1]{{#1}}
\newcommand{\ExtensionTok}[1]{{#1}}
\newcommand{\PreprocessorTok}[1]{\textcolor[rgb]{0.56,0.35,0.01}{\textit{{#1}}}}
\newcommand{\AttributeTok}[1]{\textcolor[rgb]{0.77,0.63,0.00}{{#1}}}
\newcommand{\RegionMarkerTok}[1]{{#1}}
\newcommand{\InformationTok}[1]{\textcolor[rgb]{0.56,0.35,0.01}{\textbf{\textit{{#1}}}}}
\newcommand{\WarningTok}[1]{\textcolor[rgb]{0.56,0.35,0.01}{\textbf{\textit{{#1}}}}}
\newcommand{\AlertTok}[1]{\textcolor[rgb]{0.94,0.16,0.16}{{#1}}}
\newcommand{\ErrorTok}[1]{\textcolor[rgb]{0.64,0.00,0.00}{\textbf{{#1}}}}
\newcommand{\NormalTok}[1]{{#1}}
\usepackage{graphicx,grffile}
\makeatletter
\def\maxwidth{\ifdim\Gin@nat@width>\linewidth\linewidth\else\Gin@nat@width\fi}
\def\maxheight{\ifdim\Gin@nat@height>\textheight\textheight\else\Gin@nat@height\fi}
\makeatother
% Scale images if necessary, so that they will not overflow the page
% margins by default, and it is still possible to overwrite the defaults
% using explicit options in \includegraphics[width, height, ...]{}
\setkeys{Gin}{width=\maxwidth,height=\maxheight,keepaspectratio}
\IfFileExists{parskip.sty}{%
\usepackage{parskip}
}{% else
\setlength{\parindent}{0pt}
\setlength{\parskip}{6pt plus 2pt minus 1pt}
}
\setlength{\emergencystretch}{3em}  % prevent overfull lines
\providecommand{\tightlist}{%
  \setlength{\itemsep}{0pt}\setlength{\parskip}{0pt}}
\setcounter{secnumdepth}{0}
% Redefines (sub)paragraphs to behave more like sections
\ifx\paragraph\undefined\else
\let\oldparagraph\paragraph
\renewcommand{\paragraph}[1]{\oldparagraph{#1}\mbox{}}
\fi
\ifx\subparagraph\undefined\else
\let\oldsubparagraph\subparagraph
\renewcommand{\subparagraph}[1]{\oldsubparagraph{#1}\mbox{}}
\fi

%%% Use protect on footnotes to avoid problems with footnotes in titles
\let\rmarkdownfootnote\footnote%
\def\footnote{\protect\rmarkdownfootnote}

%%% Change title format to be more compact
\usepackage{titling}

% Create subtitle command for use in maketitle
\providecommand{\subtitle}[1]{
  \posttitle{
    \begin{center}\large#1\end{center}
    }
}

\setlength{\droptitle}{-2em}

  \title{WEEK 7: SELF ASSESSMENT 3 SOLUTIONS - Shrikanth Mahale}
    \pretitle{\vspace{\droptitle}\centering\huge}
  \posttitle{\par}
    \author{}
    \preauthor{}\postauthor{}
    \date{}
    \predate{}\postdate{}
  

\begin{document}
\maketitle

Question 1: The purpose of this self-assessment is to implement
performance metrics and understand how the given stocks perform with
respect to the market, and which of the two stocks is a better
investment to make. Follow the steps below, and answer the questions
outside the code-block, in the R Markdown File in 1-2 sentences. Knit
your code to HTML using R Studio for a more presentable format and
submit the HTML file. Use echo = TRUE while knitting so that the code is
seen. a. Load the ‘Return\_Dataset.csv (Links to an external site.)’
file into your R Studio and print the first 10 rows. (1 point) Note: The
data contains monthly returns of-

--\textgreater{} Risk Free Asset (RF) --\textgreater{} Market (Mkt)
--\textgreater{} United Parcel Service (UPS) --\textgreater{} KO (Coca
Cola)

\begin{Shaded}
\begin{Highlighting}[]
\NormalTok{returns <-}\StringTok{ }\KeywordTok{read.csv}\NormalTok{(}\StringTok{"Return_Dataset.csv"}\NormalTok{)}
\KeywordTok{head}\NormalTok{(returns,}\DecValTok{10}\NormalTok{)}
\end{Highlighting}
\end{Shaded}

\begin{verbatim}
##          Date     RF          UPS           KO     Mkt
## 1  2019-08-01 0.0016 -0.006779911  0.045791340 -0.0242
## 2  2019-07-01 0.0019  0.156870364  0.041650737  0.0138
## 3  2019-06-01 0.0018  0.122018241  0.036433906  0.0711
## 4  2019-05-01 0.0021 -0.125211831  0.001426832 -0.0673
## 5  2019-04-01 0.0021 -0.049400387  0.056087991  0.0417
## 6  2019-03-01 0.0019  0.022872350  0.033524475  0.0129
## 7  2019-02-01 0.0018  0.045540711 -0.057968067  0.0358
## 8  2019-01-01 0.0021  0.080693216  0.016473177  0.0862
## 9  2018-12-01 0.0019 -0.147018565 -0.053086962 -0.0936
## 10 2018-11-01 0.0018  0.082128798  0.052631691  0.0187
\end{verbatim}

\begin{Shaded}
\begin{Highlighting}[]
\CommentTok{# fetchng libraries}
\KeywordTok{library}\NormalTok{(PerformanceAnalytics)}
\end{Highlighting}
\end{Shaded}

\begin{verbatim}
## Loading required package: xts
\end{verbatim}

\begin{verbatim}
## Loading required package: zoo
\end{verbatim}

\begin{verbatim}
## 
## Attaching package: 'zoo'
\end{verbatim}

\begin{verbatim}
## The following objects are masked from 'package:base':
## 
##     as.Date, as.Date.numeric
\end{verbatim}

\begin{verbatim}
## 
## Attaching package: 'PerformanceAnalytics'
\end{verbatim}

\begin{verbatim}
## The following object is masked from 'package:graphics':
## 
##     legend
\end{verbatim}

\begin{Shaded}
\begin{Highlighting}[]
\KeywordTok{library}\NormalTok{(xts)}
\KeywordTok{library}\NormalTok{(lubridate)}
\end{Highlighting}
\end{Shaded}

\begin{verbatim}
## Warning: package 'lubridate' was built under R version 3.4.4
\end{verbatim}

\begin{verbatim}
## 
## Attaching package: 'lubridate'
\end{verbatim}

\begin{verbatim}
## The following object is masked from 'package:base':
## 
##     date
\end{verbatim}

\begin{enumerate}
\def\labelenumi{\alph{enumi}.}
\setcounter{enumi}{1}
\tightlist
\item
  Plot and print the cumulative returns of the returns across time.
  Eventually, by seeing the graph, do UPS or KO outperform the Market?
  Explain. (2 points) Note: The graph should include the legend to
  evidently see a comparison. Hint: Use POSIXct for converting date into
  a time-series object.
\end{enumerate}

\begin{Shaded}
\begin{Highlighting}[]
\NormalTok{returns$Date <-}\StringTok{ }\KeywordTok{as.POSIXct}\NormalTok{(returns$Date, }\DataTypeTok{format=}\StringTok{"%Y-%m-%d"}\NormalTok{)}
\NormalTok{returns2<-}\StringTok{ }\NormalTok{returns[}\KeywordTok{order}\NormalTok{(returns$Date),]}
\NormalTok{All.dat <-}\StringTok{ }\KeywordTok{xts}\NormalTok{(returns2[,-}\DecValTok{1}\NormalTok{],}\DataTypeTok{order.by =} \NormalTok{returns2[,}\DecValTok{1}\NormalTok{],)}
\KeywordTok{Return.cumulative}\NormalTok{(All.dat,}\DataTypeTok{geometric =} \OtherTok{TRUE}\NormalTok{)}
\end{Highlighting}
\end{Shaded}

\begin{verbatim}
##                           RF       UPS        KO       Mkt
## Cumulative Return 0.04421815 0.2953886 0.4606529 0.5123362
\end{verbatim}

\begin{Shaded}
\begin{Highlighting}[]
\KeywordTok{chart.CumReturns}\NormalTok{(All.dat[, }\KeywordTok{c}\NormalTok{(}\DecValTok{2}\NormalTok{,}\DecValTok{3}\NormalTok{,}\DecValTok{4}\NormalTok{)],}\DataTypeTok{wealth.index =} \OtherTok{FALSE}\NormalTok{, }\DataTypeTok{geometric =} \OtherTok{TRUE}\NormalTok{,}\DataTypeTok{main =} \StringTok{"Returns Comparison: Mkt/KO/UPS"}\NormalTok{, }\DataTypeTok{legend.loc=}\StringTok{"topleft"}\NormalTok{,)}
\end{Highlighting}
\end{Shaded}

\includegraphics{sa-3_files/figure-latex/1b-1.pdf} c. Calculate and
print the Sharpe Ratio and Treynor Ratio of UPS and KO. Which of the two
is a better investment to make? Explain. (Use the StdDev Sharpe Ratio to
compare) (3 points) Note: All the output can be printed on to the
console, and the answers to the questions can be written outside the
code-block.

\begin{Shaded}
\begin{Highlighting}[]
\KeywordTok{SharpeRatio}\NormalTok{(All.dat$UPS,All.dat$RF)}
\end{Highlighting}
\end{Shaded}

\begin{verbatim}
##                                        UPS
## StdDev Sharpe (Rf=0.1%, p=95%): 0.10408624
## VaR Sharpe (Rf=0.1%, p=95%):    0.06470290
## ES Sharpe (Rf=0.1%, p=95%):     0.04710351
\end{verbatim}

\begin{Shaded}
\begin{Highlighting}[]
\KeywordTok{SharpeRatio}\NormalTok{(All.dat$KO,All.dat$RF)}
\end{Highlighting}
\end{Shaded}

\begin{verbatim}
##                                        KO
## StdDev Sharpe (Rf=0.1%, p=95%): 0.2284131
## VaR Sharpe (Rf=0.1%, p=95%):    0.1521215
## ES Sharpe (Rf=0.1%, p=95%):     0.1128447
\end{verbatim}

\begin{Shaded}
\begin{Highlighting}[]
\NormalTok{UPS_Treynor=}\StringTok{ }\KeywordTok{TreynorRatio}\NormalTok{(All.dat$UPS,All.dat$Mkt,All.dat$RF)}
\NormalTok{KO_Treynor=}\KeywordTok{TreynorRatio}\NormalTok{(All.dat$KO,All.dat$Mkt,All.dat$RF)}
\KeywordTok{print}\NormalTok{(}\KeywordTok{paste0}\NormalTok{(}\StringTok{'UPS Treynor Ratio is '}\NormalTok{,UPS_Treynor))}
\end{Highlighting}
\end{Shaded}

\begin{verbatim}
## [1] "UPS Treynor Ratio is 0.0463595502485035"
\end{verbatim}

\begin{Shaded}
\begin{Highlighting}[]
\KeywordTok{print}\NormalTok{(}\KeywordTok{paste0}\NormalTok{(}\StringTok{'KO Treynor Ratio is '}\NormalTok{,KO_Treynor))}
\end{Highlighting}
\end{Shaded}

\begin{verbatim}
## [1] "KO Treynor Ratio is 0.444949831873279"
\end{verbatim}

KO Sharpe Ratio and Treynor Ratio are higher than UPS. Also charts in 1b
reflect similar results. So KO is a better investment.

Question 2: What are the different types of market efficiency and what
does each type signify? (2 points)

Solution: Market efficiencies are of 3 types:

Weak form: This states that information that we find in past prices and
past volumes will not help us predict future returns on stocks.

Semi Strong Form: This states that all information that is publicly
available has already been reflected in the stock price and will not
help us predict future returns. Essentially suggesting that the news
headlines should be on stock / security price already.

Strong form: This notes that this is expressed in stock price as long as
information is known by at least one person. Thus the prices of
protection represent all the public and private knowledge.

Question 3: Name and briefly explain 4 types of behavioral biases? (1-2
lines) (2 points)

Solution: Below are the 4 behavioral biases investors have and they tend
to be correlated:

Overconfidence: Tendency to overstate one's capacity

Loss aversion: Individuals ' propensity to look for dignity and avoid
mistakes in their decisions

Recency effect: When people make decisions, they tend to overemphasize
recent information and not the whole range of information that could
lead us to extrapolate recent performance in the future.

Anchoring: Individuals tend to act on the basis of a single fact or
figure which should have little influence on their decision, while
ignoring more important information


\end{document}
